\documentclass{article}
    \usepackage{fullpage}
    \usepackage{amsmath}
    \usepackage{amssymb}
    \usepackage[utf8]{inputenc}
    \usepackage[T1]{fontenc}
    \textheight=10in
    \pagestyle{empty}
    \raggedright
\usepackage{graphicx} % Required for inserting images

\begin{document}

\vspace*{-10pt}
\begin{center}
	{\Huge \scshape {Mitochondria}}\\
    \textit{This is AI generated data. Use literature to confirm 'facts'.}\\
\end{center}


\section{A brief summary}


Mitochondria are organelles found in the cells of all eukaryotic organisms. They serve as an important source of energy for the cell by generating adenosine triphosphate (ATP), the main energy molecule of the cell. Mitochondria contain their own DNA and many metabolic pathways like the Kreb’s cycle and the electron transport chain. Mitochondria also play critical roles in apoptosis, calcium signaling, and certain other cellular processes.\\
\vspace{1mm}

\section{Literature}
title - \textit{authors}\\
\vspace{1mm}

    \section{Function}
    

The mitochondria, also referred to as the powerhouse of the cell, is a specialized organelle within cells that is responsible for generating energy in the form of adenosine triphosphate (ATP), which powers the cell. The mitochondria takes the energy from nutrients, such as carbohydrates and fats, and breaks them down into useable energy that can be used by the cell. This process is known as cellular respiration and is essential in providing energy for the cell to carry out its activities.\\
    \vspace{3mm}
        \textbf{title} - \textit{authors}\\
    \section{Evolution}
    

The current understanding of the evolution of mitochondria suggests that they evolved from an ancestor bacterium that was engulfed by a larger cell around 1.5 to 2 billion years ago. This ancestral bacterium had already developed the ability to produce energy from oxidative phosphorylation, a process that’s still crucial for all life today. Within the new environment of the larger cell, the ancestral mitochondrion was able to grow, reproduce itself, and become a permanent part of its new environment, giving both entities a survival advantage. Throughout the course of evolution, the mitochondrial genome has shrunk in size and has come to depend more and more on its host. However, the mitochondria are still prominent today and are essential for life throughout the animal and plant kingdoms.\\
    \vspace{3mm}
        \textbf{title} - \textit{authors}\\
\end{document}